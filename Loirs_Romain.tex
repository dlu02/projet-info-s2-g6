\documentclass{article}

\usepackage[latin1]{inputenc}
\usepackage[T1]{fontenc}
\usepackage[francais]{babel}
\usepackage[utf8]{inputenc}   
\usepackage{geometry}
\usepackage{xcolor}
\usepackage{fancyhdr}

\geometry{hmargin=2cm,vmargin=2.5cm}

\title{Projet Informatique}
 \author{Romain LOIRS}
\date{24 mars 2020}

\pagestyle{fancy} 
\lhead{Loirs Romain}
\chead{Projet Informatique} 
\rhead{ENSIIE} 
\lfoot{} 
\cfoot{} 
\rfoot{} 
\renewcommand{\headrulewidth}{0.4pt}
\renewcommand{\footrulewidth}{0.4pt} 

\begin{document}

\maketitle
\tableofcontents 

\newpage

Ce fichier a \'et\'e \'ecrit sans aucune relecture pr\'ealable

Pour commenc\'e j'ai c\'ee ce fichier latex pour vous expliquer touts les fichiers que j'ai envoy\'e (Comme cela j'aurais une trace pour un rendu plus officiel.) et parce que sur messenger cela aurait fait (un peu) long.

\paragraph{}
Pour les cartes, la structure est dans le fichier carte.h (j'ai mis toutes les structures dans les fichiers.h pour plus de lisibilit\'e pour vous). j'ai choisis d'utiliser un maximum des types \'enum\'er\'es pour les noms des cartes et les types car manipuler des entiers est plus simples pour les tests qu'une chaine de caract\'res.  
Alors comme vous avez pu le remarquer, j'ai r\'eussi \'a compresser les informations des points d\'eveloppement durable, energie , durabilit\'e et le cout dans un code que j'ai appel\'e (sobrement) code. 
Le principe est simple, j'utilise un long \` a la place d'un int ( en terme de place m\'emoire c'est la m\^eme taile mais comme j'ai long \` a la place de quatre int -> gains de place fou sur une carte puis sur le deck).  Le principe de codage est simple. Un int va jursq\'ue 32600 environ alors qu'un long va jusque 2000000000 environ. Ainsi, pour le cout, comme c'est la seul information pour les cartes de type personnel et action, vous faites code=cout et c'est simple \` a manipuler. Pour les cartes \'el\`eve c\'est la que cela devient interessant. un  long vas jusque 2000000000 et des brouettes. Je n'utilise pas le 2, donc je peux utiliser 9 chiffres. (cela tombe bien j'ai  3 types de points \`a ins\'erer) donc j'utilise les 3 premiers chiffres pour un point, les trois suivants pour l'autre (je vous donnerais un exemples... pas d'inqui\'etude) Par cons\'equent, on ne pourra pas aller à plus de 999 points pour chaque type de points. Mais bon, j'ai regardé les r\'egles, je ne pense pas qu'on tapera dans les bornes.

\paragraph{}
Pour le deck, j'ai utilis\'e un format de liste chain\'e (\# optimsation de la m\'emoire) ou chaque maillon contient une carte... cf structure.h et structure.c
J'ai repris les fonctions de Burel  et Y ( merci mes vieux TP et Damien) mais malheureusement les test d'\'egalit\'e entre des entiers et des cartes c'est pas la pareil. toutes les fonctions que je vous ai envoy\'e dans   le fichier fonctionnent. J'ai ajout\'e un \'element, retirer le premier \'element ins\'er\'ee et la sacrosainte fonction dek\_print. Il me reste retirer le dernier \'element ins\'er\'ee dans le deck et plein d'autres focntions (si en lisant les r\'egles vous trouverez des focntions, faites vous plaisir)

\paragraph{}

J'ai fais un makefile (mais pas pour le projet en entier->deception) pour compiler un fichier test (pour voir si mes focntions fonctionnent - la base quoi). Je pense qu'il ne vous sera pas utile mais bon, il vaut mieux tout partager (ahhh le communisme...)

Si vous trouvez des id\'ees d'optimisation ou des fonctionnalit\'s dont vous aurez besoin, dites le moi.

\end{document}